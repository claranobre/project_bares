\section*{\#\#\+Trabalho de Estrutura da Dados I\#\#}

Autora\+: Ana Clara (github.\+com/claranobre)

\section*{$\ast$$\ast$ P\+R\+O\+J\+E\+T\+O B.\+A.\+R.\+E.\+S. $\ast$$\ast$}

\subparagraph*{O projeto Basic Arithmetic Expression Evaluator Based On Stacks (B.\+A.\+R.\+E.\+S.) foi criado com a intenção de abordar os assuntos vistos em sala de aula.}

\subparagraph*{Proporcionando o cálculo de operações aritméticas utilizando o método infixo e pósfixo para leitura em fila e empilhamento de dados. Servirá também para verificar se as expressões apresentam os seguintes erros\+:}

{\itshape Constante numerica invalida,Falta operando, Operando inválido, Operador inválido, Falta operador, Fechamento de escopo inválido, Escopo aberto, Divisão por zero.}

Obs.\+: Esse projeto foi executado em dupla no semestre 2015.\+1 na disciplina de Laboratorio de Programação I, portanto o escopo do projeto é uma reciclagem do meu próprio projeto e de minha dupla na época.

\section*{$\ast$$\ast$\+Execução$\ast$$\ast$}

\subparagraph*{O usuário deve entrar no diretório do projeto \char`\"{}\+L\+P\+\_\+\+Bares\char`\"{} pelo terminal Linux ou Terminal simulador(\+C\+Y\+G\+W\+I\+N) do Windows}

\subparagraph*{Estando dentro do diretório o usuário deve escrever o comando \char`\"{}make\char`\"{}}

\subparagraph*{Ao terminar o comando será criado o objeto executável denominado \char`\"{}bares\char`\"{}}

\subparagraph*{Para executá-\/lo o usuário precisa digitar \char`\"{}./bares\char`\"{}}

\subparagraph*{O programa irá ler o documento \char`\"{}expressions.\+txt\char`\"{} e irá retornar os resultados das expressões listadas, ou erros que forem encontrados em cada expressão.}

\subparagraph*{Ao terminar de executar o programa, se caso o usuário quiser testar alguma modificação no código original, na compilação após essa alteração, antes de dar o comando \char`\"{}make\char`\"{} será necessário excluir com o comando {\itshape rm bares} o objeto da compilação anterior}

\subparagraph*{Caso o usuário só quiser modificar o arquivo \char`\"{}expressions.\+txt\char`\"{} para testar outras expressões, ele só precisará ir no diretório {\itshape src} e editar o arquivo, salvar e executar o programa (./bares). }

\section*{$\ast$$\ast$\+Verificação de Vazamento de Memória$\ast$$\ast$ }

\subparagraph*{Para verificarmos se o nosso algoritmo está com vazamento de memória de dados, utilizamos a ferramenta Valgrind para teste}

\subparagraph*{O usuário após compilar e criar o objeto deve escrever no Terminal\+:}

{\itshape valgrind --leak-\/check=full ./bares}

\section*{$\ast$$\ast$\+Documentação$\ast$$\ast$}

\subparagraph*{Foi utilizada a ferramenta Doxygen para auxiliar na documentação da execução desse projeto, para o usuário visualizar é necessário entrar no diretório \char`\"{}html\char`\"{} e acessar o arquivo index.\+html, logo a documentação será aberta em modo offline em seu navegador padrão}